\documentclass[12pt]{article}
\usepackage{fontspec}
\usepackage[utf8]{inputenc}
\setmainfont{Bodoni 72 Book}
\usepackage[paperwidth=11in,paperheight=17in,margin=1in,headheight=0.0in,footskip=0.5in,includehead,includefoot,portrait]{geometry}
\usepackage[absolute]{textpos}
\TPGrid[0.5in, 0.25in]{23}{24}
\parindent=0pt
\parskip=12pt
\usepackage{nopageno}
\usepackage{graphicx}
\graphicspath{ {./images/} }
\usepackage{amsmath}
\usepackage{hyperref}
\usepackage{tikz}
\newcommand*\circled[1]{\tikz[baseline=(char.base)]{
            \node[shape=circle,draw,inner sep=1pt] (char) {#1};}}

\begin{document}

\vspace*{1\baselineskip}

\begingroup
\begin{center}
\huge NOTES TO THE INTERPRETERS
\end{center}
\endgroup

\begingroup
\textbf{General: \circled{1}} After temporary \textbf{accidentals}, cancellation marks are printed also in the following measure ( for notes in the same octave ) and, in the same measure, for notes in other octaves, but they are printed again if the same note appears later in the same measure, except if the note is immediately repeated. \\
\textbf{\circled{2} Microtones} present in this score are \textbf{quarter-tones} and \textbf{rational intervals}. \\
\circled{1.} \includegraphics[scale=0.05]{qt_flat.png} indicates a \textbf{quarter-tone flat}. \\
\circled{2.} \includegraphics[scale=0.05]{qt_sharp.png} indicates a \textbf{quarter-tone sharp}.\\
\circled{3.} \textbf{Justly tuned intervals} are indicated by the use of \textbf{Helmholtz-Ellis accidental system} combined with \textbf{cent deviations from equal temperament} for use with an electronic tuner. When no example pitch is given with the cent deviation, the mark is a deviation of the nearest ``standard” accidental. In the absence of electronic tuners, approximations of these deviations are acceptable. Ratios are taken from a common fundamental to be shared between two interpreters who tune together. So, when playing rationally tuned intervals, a note reading \textbf{``( partial number )° / \\ ( fundamental ) with ( other instrument )."} \\ 
\textbf{\circled{3} Dynamics} enclosed in quotation marks refer to \textbf{physical intensity} rather than sounding volume. \\
\textbf{\circled{4} Time signatures whose denominator is note a power of two} such as \textbf{4/12} are to be understood as \textbf{prolated subdivisions of the whole note.} In this case, the denominator indicates a twelfth of a whole note, or, a triplet eighth note. In this idiom, \textbf{tuplet brackets with the right side open} indicate the prolation of the note alone, rather than the duration of the full tuplet. \\
\textbf{\circled{5} Grace notes on the beat} should be attacked and played as quickly as possible with the note to which they are attached. \\
\textbf{\circled{6} Grace figures enclosed in duration brackets}, as below: \\
\begin{center}
\includegraphics[scale=0.3]{grace_figures.png} \\
\end{center}
should be freely scaled to the time indicated by the bracket. In the case above, an aleatoric assortment of the instruments' highest possible notes should be rapidly played within the duration of seven eighth notes.
\endgroup 

%\pagebreak

\begingroup
\textbf{Electronics: \circled{1} The instruments should be amplified} to balance with fixed media electronics, and to aid in the production of \textbf{audible combination tones} when rational intervals are played. To this end, the \textbf{clarinet} and \textbf{violin} are sent as \textbf{mono signals} to the \textbf{left channel} of a \textbf{stereo speaker system}, and the \textbf{flute} and \textbf{cello} are sent as \textbf{mono signals} to the \textbf{right channel}. The \textbf{accordion} is sent as a \textbf{stereo signal} to \textbf{both channels}. \\
\textbf{\circled{2} Fixed media} files are played at the start of various measures, indicated in the score by \textbf{boxed notes} above the top staff. The interpreters need not syncronise their playing with the contents of the files. \\
\endgroup 

\begingroup
\textbf{Woodwinds: \circled{1} Throat-screams} while playing are occasionally prescribed. This technique is performed as a close-mouthed scream held in the back of the throat. This scream can follow the breath, and \textbf{need not be totally continuous}. It is acceptable if and expected that the performance of this technique affects the playing of the instruments. 
\endgroup 

\begingroup
\textbf{Flute: \circled{1} The flutist plays} a \textbf{piccolo} and an \textbf{alto flute}. \\
\textbf{\circled{2} Consonant-vowel pairings} are occasionally indicated beneath the staff to perform during the attack of the note to which they are attached. \\
\textbf{\circled{3} Angle of the head joint} in relation to the mouth is indicated with the following symbols: \\ 
\circled{1.} \includegraphics[scale=0.05]{flute_normale.png} indicates \textbf{ordinary head joint angle}. \\
\circled{2.} \includegraphics[scale=0.05]{flute_one_hundred_thirty_five_degrees.png} indicates to turn the head joint \textbf{45 degrees away} from the mouth.\\
\circled{3.} \includegraphics[scale=0.05]{flute_forty_five_degrees.png} indicates to turn the head joint \textbf{45 degrees towards} the mouth.\\
\circled{4.} \includegraphics[scale=0.05]{flute_ninety_degrees.png} indicates to turn the head joint until it is \textbf{directly against the mouth}.\\
 
\pagebreak

\textbf{\circled{5} One multiphonic} is used in this piece, played with the alto flute. The first time it appears, it is indicated using a \textbf{fingering diagram}, as below: \\
\begin{center}
\includegraphics[scale=0.25]{flute_multiphonic_fingering.png} \\
\end{center}
Afterwards, it is indicated using an \textbf{M enclosed in parentheses}, as below: \\
\begin{center}
\includegraphics[scale=0.2]{flute_multiphonic_symbol.png} \\
\end{center}
\endgroup 

\begingroup
\textbf{Clarinet: \circled{1} The clarinetist plays} a \textbf{soprano clarinet in B-flat} and a \textbf{bass clarinet}. \\
\textbf{\circled{2}} The clarinetist should be equipped with a thin, tall \textbf{traffic cone} to insert into the bell of the clarinet in order to \textbf{deepen and obscure} the instruments' pitch profile. \\
\textbf{\circled{3} Overblowing} is occasionally abbreviated as \textbf{o.b.} \\
\textbf{\circled{4} Circled numbers above the staff}, as below: 
\begin{center}
\includegraphics[scale=0.25]{color_fingerings.png} \\
\end{center}
indicate \textbf{alternative fingerings} of the written pitch. These alternative fingerings vary the \textbf{intonation} and \textbf{timbre} of the note. Though the clarinetist is at liberty to choose their own fingerings to achieve this effect, some suggestions are provided below. \\
\begin{center}
\textbf{Soprano:}
\end{center}
\textbf{A:} \includegraphics[scale=0.12]{soprano_clarinet_a_fingerings.png} \hspace{6mm} \textbf{F-sharp:} \includegraphics[scale=0.1]{soprano_clarinet_f_sharp_fingerings.png} \\ \\ \\
\begin{center}
\textbf{Bass:}
\end{center}
\textbf{G-sharp:} \includegraphics[scale=0.1]{bass_clarinet_g_sharp_fingerings.png} \hspace{6mm}
\textbf{A:} \includegraphics[scale=0.11]{bass_clarinet_a_fingerings_1.png}\includegraphics[scale=0.11]{bass_clarinet_a_fingerings_2.png} \hspace{6mm}
\textbf{B:} \includegraphics[scale=0.11]{bass_clarinet_b_fingerings_1.png}\includegraphics[scale=0.1]{bass_clarinet_b_fingerings_2.png} \hspace{6mm}
\textbf{D-sharp:} \includegraphics[scale=0.13]{bass_clarinet_d_sharp_fingerings.png} \\ \\
\endgroup 

\begingroup
\textbf{Accordion: \circled{1} The upper staff} corresponds to the \textbf{right-hand manual}, and the \textbf{lower staff} corresponds to the \textbf{left-hand manual}. \\
\textbf{\circled{2} Portamento glissandi} are occasionally performed by \textbf{gradually closing the reed} through the \textbf{slow release of a button or key.} \\
\textbf{\circled{3} The air button} is sometimes partially depressed with the left hand \textbf{while playing}, indicated with the direction \textbf{``( fraction ) air"}. \\
\textbf{\circled{4}} At measure \textbf{56}, the text direction \textbf{bellows} indicates to place the hand \textbf{in between the extended bellows} and rapidly move the hand back and forth, creating a sound like a crow flapping its wings. 
\endgroup

\begingroup
\textbf{Strings: \circled{1} Levels of spazzolato} ( abbreviated \textbf{``spz."} ) are used to prescribe amounts of \textbf{vertical bow motion}, wherein \textbf{norm.} indicates \textbf{no vertical bow motion}, \textbf{spz.} indicates \textbf{no horizontal bow motion}, and fractional spazzolato such as \textbf{1/2 spz.} or \textbf{3/4 spz.} indicate approximate amounts of vertical bow motion between the two, resulting in variably angled diagonal bowing. \\
\textbf{\circled{2} Bow-tip angle } is prescribed using the symbols below: \\
\circled{1.} \includegraphics[scale=0.05]{bow_normale.png} Point the tip of the bow \textbf{perpendicular} to the instrument. \\
\circled{2.} \includegraphics[scale=0.05]{bow_one_hundred_thirty_five_degrees.png} Point the tip of the bow \textbf{towards the bottom of the instrument}.\\
\circled{3.} \includegraphics[scale=0.05]{bow_forty_five_degrees.png} Point the tip of the bow \textbf{towards the top of the instrument}.\\
\circled{4.} \includegraphics[scale=0.05]{bow_up.png} Point the tip of the bow \textbf{directly towards the scroll}, parallel with the strings.\\
\endgroup 

\begingroup
\textbf{Cello: \circled{1} Wavy glissandi} indicate the relative \textbf{speed} and \textbf{width} of a vibrato. 
\endgroup 

\end{document}